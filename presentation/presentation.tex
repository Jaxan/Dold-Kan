\documentclass[14pt]{beamer}

% beamer definieert 'definition' al, maar dan engels :(
% fix van:
% http://tex.stackexchange.com/questions/38392/how-to-rename-theorem-or-lemma-in-beamer-to-another-language
\usepackage[dutch]{babel}
\uselanguage{dutch}
\languagepath{dutch}
\deftranslation[to=dutch]{Definition}{Definitie}

\usepackage{graphicx}
\usepackage{float}
\usepackage{amssymb}
\usepackage{color}
\usepackage{listings}

\newcommand{\id}{\text{id}}
\newcommand{\N}{\mathbb{N}}
\newcommand{\Z}{\mathbb{Z}}
\newcommand{\R}{\mathbb{R}}
\newcommand{\cat}[1]{\mathbf{#1}}
\newcommand{\Ch}[1]{\mathbf{Ch}(#1)}
\newcommand{\Hom}[3]{\mathbf{Hom}_{#1}(#2, #3)}

\newcommand{\iso}{\cong}
\newcommand{\tot}[1]{\xrightarrow{\,\,{#1}\,\,}}
\newcommand{\eps}{\varepsilon}
\newcommand{\I}{\,\mid\,}
\newcommand{\then}{\Rightarrow}
\newcommand{\inject}{\hookrightarrow}
\newcommand{\del}{\partial}
\newcommand{\nsubgrp}{\trianglelefteq}

% relative to the one who includes us :(
\graphicspath{ {../images/} }

\newcommand{\todo}[1]{
	\addcontentsline{tdo}{todo}{\protect{#1}}
	$\ast$ \marginpar{\tiny $\ast$ #1}
}
\makeatletter
	\newcommand \listoftodos{\section*{Todo list} \@starttoc{tdo}}
	\newcommand\l@todo[2]{
		\par\noindent \textit{#2}, \parbox{10cm}{#1}\par
	}
\makeatother


\title{Dold-Kan correspondentie
	\huge $$ \Ch{\cat{Ab}} \simeq \cat{sAb} $$}
\author{Joshua Moerman}
\institute[Radboud Universiteit Nijmegen]{Begeleid door Moritz Groth}
\date{}

\begin{document}


\begin{frame}
	\titlepage
\end{frame}


\begin{frame}
	\frametitle{Wat is $\Ch{\cat{Ab}}$?}
	\begin{definition}
	Een \emph{ketencomplex} $C$ bestaat uit abelse groepen met groepshomomorfisme:
	$$ \cdots \to C_4 \to^{\del_3} C_3 \to^{\del_2} C_2 \to^{\del_1} C_1 \to^{\del_0} C_0 $$

	zodat $\del_n \circ \del_{n+1} = 0$ voor alle $n \geq 1$.
	\end{definition}
\end{frame}


\begin{frame}
	\frametitle{Voorbeeld}
	Bekijk $\Delta^n \to X$, dwz...
\end{frame}


\begin{frame}
	\frametitle{Is $\Ch{\cat{Ab}}$ interessant?}
	Gegeven een ketencomplex $C$:
	$$ \cdots \to C_4 \to^{\del_3} C_3 \to^{\del_2} C_2 \to^{\del_1} C_1 \to^{\del_0} C_0 $$
	met $\del_n \circ \del_{n+1} = 0$
	\bigskip
	
	Dan geldt $im(\del_{n+1}) \trianglelefteq ker(\del_n)$

	Definieer: $H_n(C) = ker(\del_n) / im(\del_{n+1})$
\end{frame}


\begin{frame}
	\frametitle{Voorbeeld}
	$ \cdots \to C_1 \to^{\del_0} C_0 $, wat is $H_1 = ker(\del_0) / im(\del_1)$?

	\begin{enumerate}
		\item Triviaal
		\item Niet triviaal
	\end{enumerate}
\end{frame}


\begin{frame}
	\frametitle{Dold-Kan Correspondentie}
	\begin{center}
	{\Large $ \Ch{\cat{Ab}} \simeq \cat{sAb} $}

	verder:
	{\Large $$ H_n(N(X)) \iso \pi_n(X) $$}
	waarbij $N : \cat{sAb} \to \Ch{\cat{Ab}}$.
	\end{center}
\end{frame}


\begin{frame}
	\begin{center}
	\Huge Vragen?
	\end{center}
\end{frame}


\end{document}
