\documentclass[14pt]{beamer}

\usepackage[dutch]{babel}

\usepackage{graphicx}
\usepackage{float}
\usepackage{amssymb}
\usepackage{color}
\usepackage{listings}

\newcommand{\id}{\text{id}}
\newcommand{\N}{\mathbb{N}}
\newcommand{\Z}{\mathbb{Z}}
\newcommand{\cat}[1]{\mathbf{#1}}
\newcommand{\eps}{\varepsilon}
\newcommand{\I}{\,\mid\,}
\newcommand{\then}{\Rightarrow}
\newcommand{\inject}{\hookrightarrow}
\newcommand{\del}{\partial}

\title{Dold-Kan correspondentie}
\author{Joshua Moerman}
\institute[Radboud Universiteit Nijmegen]{Begeleid door Moritz Groth}
\date{}

\begin{document}

\begin{frame}
  \titlepage
\end{frame}

\begin{frame}
\frametitle{Dold-Kan Correspondentie}
\huge $$ \cat{Ch(Ab)} \simeq \cat{sAb} $$
\end{frame}

\section{Ketencomplex}
\begin{frame}
\frametitle{Ketencomplex}
\begin{definition}
	Een \emph{ketencomplex} $C$ bestaat uit abelse groepen $C_n$ en homomorfismes $\del_n : C_{n+1} \to C_n$, zodat $\del_n \circ \del_{n+1} = 0$ voor alle $n \in \N$.
\end{definition}
\pause
\bigskip
Met andere woorden:
$$ \cdots \to C_4 \to C_3 \to C_2 \to C_1 \to C_0 $$
\end{frame}

\begin{frame}
Uit $\del_n \circ \del_{n+1} = 0$ volgt $im(\del_{n+1}) \trianglelefteq ker(\del_n)$
\pause
Definieer: $H_n(C) = ker(\del_n) / im(\del_{n+1})$
\end{frame}

\begin{frame}
\begin{center}
\Huge Vragen?
\end{center}
\end{frame}

\end{document}