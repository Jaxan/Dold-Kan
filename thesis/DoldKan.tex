\documentclass[12pt]{amsproc}

% a la fullpage
\usepackage{geometry}
\geometry{a4paper}
\geometry{twoside=false}

% Activate to begin paragraphs with an empty line rather than an indent
\usepackage[parfill]{parskip}
\setlength{\marginparwidth}{2cm}

\newtheorem{theorem}{Theorem}[section]
\newtheorem{definition}[theorem]{Definition}
\newtheorem{lemma}[theorem]{Lemma}

\usepackage{graphicx}
\usepackage{float}
\usepackage{amssymb}
\usepackage{color}
\usepackage{listings}

%\newcommand{\id}{\text{id}}
\newcommand{\N}{\mathbb{N}}
\newcommand{\Z}{\mathbb{Z}}
\newcommand{\R}{\mathbb{R}}
\newcommand{\cat}[1]{\mathbf{#1}}
\newcommand{\Ab}{\cat{Ab}}
\newcommand{\sAb}{\cat{sAb}}
\newcommand{\Set}{\cat{Set}}
\newcommand{\Ch}[1]{\mathbf{Ch}(#1)}
\newcommand{\Hom}[3]{\mathbf{Hom}_{#1}(#2, #3)}
\newcommand{\id}{\mathbf{id}}

\newcommand{\iso}{\cong}
\newcommand{\tot}[1]{\xrightarrow{\,\,{#1}\,\,}}
\newcommand{\eps}{\varepsilon}
\newcommand{\I}{\,\mid\,}
\newcommand{\then}{\Rightarrow}
\newcommand{\inject}{\hookrightarrow}
\newcommand{\del}{\partial}
\newcommand{\nsubgrp}{\trianglelefteq}

% relative to the one who includes us :(
\graphicspath{ {../images/} }

\newcommand{\todo}[1]{
	\addcontentsline{tdo}{todo}{\protect{#1}}
	$\ast$ \marginpar{\tiny $\ast$ #1}
}
\makeatletter
	\newcommand \listoftodos{\section*{Todo list} \@starttoc{tdo}}
	\newcommand\l@todo[2]{
		\par\noindent \textit{#2}, \parbox{10cm}{#1}\par
	}
\makeatother


\title{Dold-Kan Correspondence}
\author{Joshua Moerman}

\begin{document}
\maketitle

\section*{Introduction}
In this thesis we will look at a correspondence which was discovered by A. Dold and D. Kan independently, hence it is called the \emph{Dold-Kan correspondence}. Abstractly it is the following equivalence of categories:
$$ \Ch{\Ab} \simeq \sAb $$
It is interesting because objects on the left hand side are considered to be algebraic of nature, whereas objects on the right are more topological. In particular this correspondence also gives a isomorphism between homology groups (on the left hand side) and homotopy groups (on the right hand side). A bit more precise:
$$ \pi_n(A) \iso H_n(N(A)) \text{ for all } n \in \N $$
where $N: \sAb \to \Ch{\Ab}$ is one half of the equivalence.

\newpage
\section{Category Theory}
\label{sec:Category Theory}
Before we will introduce the two categories $\Ch{\Ab}$ and $\sAb$, we will first look at some basic category theory. If one is already familier with these concepts, he or she can skip this section. We will introduce the notions of categories, functors, isomorphims, natural transformations, equivalences (between categories) and adjunctions.

\subsection{Categories}
\begin{definition}
	A \emph{category} $\cat{C}$ consists of a collection \emph{objects}, and for each two objects $A$ and $B$ in $\cat{C}$ there is a (possibly empty) \emph{set of maps} (or arrows) from $A$ to $B$, notated as $\Hom{\cat{C}}{A}{B}$, such that:
	\begin{itemize}
		\item \emph{(Identity)}
			$\id_A \in \Hom{\cat{C}}{A}{A}$ for all $A$ in $\cat{C}$,
		\item \emph{(Composition)}
			for any $f \in \Hom{\cat{C}}{A}{B}$ and $g \in \Hom{\cat{C}}{B}{C}$ we have $g \circ f \in \Hom{\cat{C}}{A}{C}$,
		\item \emph{(Associativity)}
			$f \circ (g \circ h) = (f \circ g) \circ h$, and
		\item \emph{(Identity law)}
			$\id_B \circ f = f = f \circ \id_A$ for all $f \in \Hom{\cat{C}}{A}{B}$.
	\end{itemize}
\end{definition}

Note that the collection of objects may be a proper class instead of a set, however we will notate $A \in \cat{C}$ if $A$ is an object of $\cat{C}$. And instead of writing $f \in \Hom{\cat{C}}{A}{B}$, we write $f: A \to B$.

As the notation suggests maps can be thought of as functions, which is also the case in many examples.

\begin{lemma}
	The category $\Set$ has a objects sets, and as maps ordinary functions. Of course we then have the identity function $\id_X(x) = x$ and composition as usual.
\end{lemma}
\begin{lemma}
	The category $\Ab$ has a objects abelian groups, and the maps between two objects are exactly the grouphomomorphisms. We know that the identity function is indeed a grouphomomorphism, and composing two grouphomomorpisms, gives indeed a new grouphomomorphism.
\end{lemma}

In fact almost any mathematical structure can be described as a category, we have: $\cat{Ring}$ for rings, $\cat{Vect}$ for $\R$-vectorspaces, $\cat{Set_{fin}}$ for finite sets, $\cat{Poset}$ for posets, etc. Of course we would also like to express relations between categories, for example every abelian group is also a set. This idea can be formulated by the notion of a functor.

\begin{definition}
	A \emph{functor} $F$ between a category $\cat{C}$ and $\cat{D}$ consists of a function $F_0$ from the objects of $\cat{C}$ to the objects of $\cat{D}$ and a function $F_1$ from maps in $\cat{C}$ to maps in $\cat{D}$, such that:
	\begin{itemize}
		\item for $f: A \to B$, we have $F_1(f): F_0(A) \to F_0(B)$,
		\item $F_1(\id_A) = \id_{F_0(A)}$ and
		\item $F_1(f \circ g) = F_1(f) \circ F_1(g)$.
	\end{itemize}
	We normally do not write the index of $F_0$ or $F_1$, instead we wrtie $F$ for both functions.
\end{definition}

\begin{lemma}
	There is a category $\cat{Cat}$ with categories as objects, and functors as maps.
\end{lemma}
\begin{proof}
	First we define the identity functor. Let $\cat{C}$ be a category, define $\id_\cat{C}(A) = A$ for any object $A \in \cat{C}$ and $\id_\cat{C}(f) = f$ for any map $f: A \to B$ in $\cat{C}$. Cleary we have $\id_\cat{C}(f) : \id_\cat{C}(A) \to \id_\cat{C}(B)$. Also $\id_\cat{C}(\id_A) = \id_A = \id_{\id_\cat{C}(A)}$ and $\id_\cat{C}(f \circ g) = f \circ g$. So indeed $\id_\cat{C}$ is a functor.

	Given a functors $F: \cat{C} \to \cat{D}$ and $G: \cat{D} \to \cat{E}$, we can define the composition $G \circ F$ on objects as $G \circ F(A) = G(F(A))$ and on maps as $G \circ F(f) = G(F(f))$. This again is a functor $G \circ F$, we will not spell out the details.

	The remaining requirements are the associativity and identity law. We also leave these to the reader.
\end{proof}

\subsection{Isomorphisms}
Given a category $\cat{C}$ and two objects $A, B \in \cat{C}$ we would like to know when those objects are regarded as the same, according to the category. This will be the case when there is an isomorphism between the two.

\begin{definition}
	A map $f: A \to B$ in a category $\cat{C}$ is an isomorphism if there is a map $g: B \to A$ such that:
	$$ f \circ g = \id_B \text{ and } g \circ f = id_A.$$
\end{definition}

Isomorphisms in $\Ab$ are exactly the isomorphisms which we know, ie. the grouphomomorphisms which are both injective and surjective.
For example the cyclic group $\Z_4$ and the klein four-group $V_4$ are not isomorphic in $\Ab$, but if we regard only the sets $\Z_4$ and $V_4$, then they are (because there is a bijection). So it is good to note that whether two objects are isomorphic  really depends on the category we are working in.

\todo{CT: Equivalence / natro}
\todo{CT: Adjunction}


\newpage
\section{Chain Complexes}
\label{sec:Chain Complexes}
\begin{definition}
	A \emph{(non-negative) chain complex} $C$ is a collection of abelian groups $C_n$ together with group homomorphisms $\del_n: C_n \to C_{n-1}$, which we call \emph{boundary homomorphisms}, such that $\del_n \circ \del_{n+1} = 0$ for all $n \in \Np$.
\end{definition}

Thus graphically a chain complex $C$ can be depicted by the following diagram:
\begin{center}
\begin{tikzpicture}
	\matrix (m) [matrix of math nodes]{
		\cdots & C_4 & C_3 & C_2  & C_1 & C_0 \\
	};
	\foreach \d/\i/\j in {5/1/2,4/2/3,3/3/4,2/4/5,1/5/6} \path[->] (m-1-\i) edge node[auto] {$ \del_\d $} (m-1-\j);
\end{tikzpicture}
\end{center}

There are many variants to this notion. For example, there are also unbounded chain complexes with an abelian group for each $n \in \Z$ instead of $\N$. In this thesis we will only need chain complexes in the sense of the definition above. Hence we will simply call them chain complexes, instead of non-negative chain complexes. Other variants can be given by taking a collection of $R$-modules instead of abelian groups. Of course not any kind of mathematical object will suffice, because we need to be able to express $\del_n \circ \del_{n+1} = 0$, so we need some kind of \emph{zero object}. We will not need this kind of generality and stick to abelian groups.

In order to organize these chain complexes in a category, we should define what the maps are. The diagram above already gives an idea for this.
\begin{definition}
	Let $C$ and $D$ be chain complexes, with boundary maps $\del^C_n$ and $\del^D_n$ respectively. A \emph{chain map} $f: C \to D$ consists of a family of maps $f_n: C_n \to D_n$, such that they commute with the boundary operators: $f_n \circ \del^C_{n+1} = \del^D_{n+1} \circ f_{n+1}$ for all $n \in \N$, i.e. the following diagram commutes:
	\begin{center}
	\begin{tikzpicture}
		\matrix (m) [matrix of math nodes]{
			\cdots & C_4 & C_3 & C_2  & C_1 & C_0 \\
			\cdots & D_4 & D_3 & D_2  & D_1 & D_0 \\
		};
		\foreach \d/\i/\j in {5/1/2,4/2/3,3/3/4,2/4/5,1/5/6} \path[->] (m-1-\i) edge node[auto] {$ \del^C_\d $} (m-1-\j);
		\foreach \d/\i/\j in {5/1/2,4/2/3,3/3/4,2/4/5,1/5/6} \path[->] (m-2-\i) edge node[auto] {$ \del^D_\d $} (m-2-\j);
		\foreach \d/\i in {4/2,3/3,2/4,1/5,0/6} \path[->] (m-1-\i) edge node[auto] {$ f_\d $} (m-2-\i);
	\end{tikzpicture}
	\end{center}
\end{definition}

Note that if we have two such chain maps $f:C \to D$ and $g:D \to E$, then the levelwise composition will give us a chain map $g \circ f: C \to D$. Also taking the identity function in each degree, gives us a chain map $\id : C \to C$. In fact, this will form a category, we will leave the details (the identity law and associativity) to the reader.

\begin{definition}
	$\Ch{\Ab}$ is the category of chain complexes with chain maps.
\end{definition}

Note that we will often drop the indices of the boundary morphisms, since it is often clear in which degree we are working. The boundary operators give rise to certain subgroups, because all groups are abelian, subgroups are normal subgroups.

\begin{definition}
	Given a chain complex $C$ we define the following subgroups:
	\begin{itemize}
		\item $Z_n(C) = \ker(\del_n: C_n \to C_{n-1}) \nsubgrp C_n$, and
		\item $Z_0(C) = C_0$, and
		\item $B_n(C) = \im(\del_{n+1}: C_{n+1} \to C_n) \nsubgrp C_n$.
	\end{itemize}
\end{definition}
\begin{lemma}
	Given a chain complex $C$ we have for all $n \in \N$:
	$$ B_n(C) \nsubgrp Z_n(C).$$
\end{lemma}
\begin{proof}
	It follows from $\del_n \circ \del_{n+1} = 0$ that $\im(\del: C_{n+1} \to C_n)$ is a subset of $\ker(\del: C_n \to C_{n-1})$. Those are exactly the abelian groups $B_n(C)$ and $Z_n(C)$, so $ B_n(C) \nsubgrp Z_n(C) $.
\end{proof}
\begin{definition}
	Given a chain complex $C$ we define the \emph{$n$-th homology group} $H_n(C)$ for each $n \in \N$ as:
	$$ H_n(C) = Z_n(C) / B_n(C).$$
\end{definition}

\todo{CC: $H_n$ as a functor} 

\subsection{The singular chain complex}
In order to see why we are interested in the construction of homology groups, we will look at an example from algebraic topology. We will see that homology gives a nice invariant for spaces. So we will form a chain complex from a topological space $X$. In order to do so, we first need some more notions.
\begin{definition}
	The topological space $\Delta^n$ is called the \emph{topological $n$-simplex} and is defined as:
	$$ \Delta^n = \{(x_0, x_1, \ldots, x_n) \in \R^{n+1} \I x_i \geq 0 \text{ and } x_0 + \ldots + x_n = 1 \}.$$
	The topology on $\Delta^n$ is the subspace topology.
\end{definition}

In particular $\Delta^0$ is simply a point, $\Delta^1$ a line and $\Delta^2$ a triangle. There are nice inclusions $\Delta^n \mono \Delta^{n+1}$ which we need later on. For any $n \in \N$ we define:
\begin{definition}
	For $i \in \{0, \ldots, n+1\}$ the $i$-th face map $\delta^i : \Delta^n \mono \Delta^{n+1}$ is defined as:
	$$ \delta^i (x_0, \ldots, x_n) = (x_0, \ldots, x_{i}, 0, x_{i+1}, \ldots, x_n) \text{ for all } x \in \Delta^n.$$
\end{definition}

For any space $X$, we will be interested in continuous maps $\sigma : \Delta^n \to X$, such a map is called a $n$-simplex. Note that if we have any continuous map $\sigma : \Delta^{n+1} \to X$ we can precompose with a face map to get $\sigma \circ \delta^i : \Delta^n \to X$, as shown in figure~\ref{fig:diagram_d}. This will be used for defining the boundary operator. We can make pictures of this, and when concerning continuous maps $\sigma : \Delta^{n+1} \to X$ we will draw the images in the space $X$, instead of functions.

\begin{figure}
	\begin{tikzpicture}
		\matrix (m) [matrix of math nodes]{
			\Delta^{n+1} & X \\
			\Delta^n & \\
		};
		\path[->]
		(m-1-1) edge node[auto] {$ \sigma $} (m-1-2)
		(m-2-1) edge node[auto] {$ \delta^i $} (m-1-1)
		(m-2-1) edge node[auto] {$ $} (m-1-2);
	\end{tikzpicture}
	\caption{The $(n+1)$-simplex $\sigma$ can be made into a $n$-simplex $\sigma \circ \delta^i$}
	\label{fig:diagram_d}
\end{figure}

\todo{Ch: Make some pictures here}

We now have enough tools to define the singular chain complex of a space $X$.

\begin{definition}
	For a topological space $X$ we define the \emph{$n$-th singular chain group} $C_n(X)$ as follows.
	$$ C_n(X) = \Z[\Hom{\cat{Top}}{\Delta^n}{X}] $$
	The boundary operator $\del : C_{n+1}(X) \to C_n(X)$ is defined on generators as:
	$$ \del(\sigma) = \sigma \circ \delta^0 - \sigma \circ \delta^1 + \ldots + (-1)^{n+1} \sigma \circ \delta^{n+1}.$$
\end{definition}

This might seem a bit complicated, but we can pictures this in an intuitive way, as in figure~\ref{fig:singular_chaincomplex3}. And we see that the boundary operators really give the boundary of an $n$-simplex. To see that this indeed is a chain complex we have to proof that the composition of two such operators is the zero map.
\begin{figure}[h!]
	\includegraphics{singular_chaincomplex3}
	\caption{The boundary of a 2-simplex}
	\label{fig:singular_chaincomplex3}
\end{figure}
\todo{CC: update picture}

\todo{Ch: Proposition: $C(X) \in \Ch{\cat{Ab}}$}
\todo{Ch: Example homology of some space}
\todo{Ch: Show that $\Ch{\Ab}$ is an ab. cat. At least show functoriality $\Hom{\Ch{\Ab}}{-}{-}$}


\newpage
\listoftodos
% \nocite{*}
% \bibliographystyle{alpha}
% \bibliography{references}	
\end{document}
