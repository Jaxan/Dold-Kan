\documentclass[12pt]{amsproc}

% a la fullpage
\usepackage{geometry}
\geometry{a4paper}
\geometry{twoside=false}

% Activate to begin paragraphs with an empty line rather than an indent
\usepackage[parfill]{parskip}
\setlength{\marginparwidth}{2cm}

\newtheorem{theorem}{Theorem}[section]
\newtheorem{definition}[theorem]{Definition}
\newtheorem{lemma}[theorem]{Lemma}

\usepackage{graphicx}
\usepackage{float}
\usepackage{amssymb}
\usepackage{color}
\usepackage{listings}

%\newcommand{\id}{\text{id}}
\newcommand{\N}{\mathbb{N}}
\newcommand{\Z}{\mathbb{Z}}
\newcommand{\R}{\mathbb{R}}
\newcommand{\cat}[1]{\mathbf{#1}}
\newcommand{\Ab}{\cat{Ab}}
\newcommand{\sAb}{\cat{sAb}}
\newcommand{\Set}{\cat{Set}}
\newcommand{\Ch}[1]{\mathbf{Ch}(#1)}
\newcommand{\Hom}[3]{\mathbf{Hom}_{#1}(#2, #3)}
\newcommand{\id}{\mathbf{id}}

\newcommand{\iso}{\cong}
\newcommand{\tot}[1]{\xrightarrow{\,\,{#1}\,\,}}
\newcommand{\eps}{\varepsilon}
\newcommand{\I}{\,\mid\,}
\newcommand{\then}{\Rightarrow}
\newcommand{\inject}{\hookrightarrow}
\newcommand{\del}{\partial}
\newcommand{\nsubgrp}{\trianglelefteq}

% relative to the one who includes us :(
\graphicspath{ {../images/} }

\newcommand{\todo}[1]{
	\addcontentsline{tdo}{todo}{\protect{#1}}
	$\ast$ \marginpar{\tiny $\ast$ #1}
}
\makeatletter
	\newcommand \listoftodos{\section*{Todo list} \@starttoc{tdo}}
	\newcommand\l@todo[2]{
		\par\noindent \textit{#2}, \parbox{10cm}{#1}\par
	}
\makeatother


\title{Dold-Kan Correspondence}
\author{Joshua Moerman}

\begin{document}
\maketitle

\section{Introduction}
In this thesis we will look at a correspondence which was discovered by A. Dold and D. Kan independently, hence it is called the \emph{Dold-Kan correspondence}. Abstractly it is the following equivalence of categories:
$$ \Ch{\cat{Ab}} \simeq \cat{sAb} $$
It is interesting because objects on the left hand side are considered to be algebraic of nature, whereas objects on the right are more topological. In particular this correspondence also gives a isomorphism between homology groups (on the left hand side) and homotopy groups (on the right hand side). A bit more precise:
$$ \pi_n(A) \iso H_n(N(A)) \text{ for all } n \in \N $$
where $N: \cat{sAb} \to \Ch{\cat{Ab}}$ is one half of the equivalence.

\section{Chain Complexes}
\begin{definition}
	A chain complex $C$ is a collection of abelian groups $C_n$ together with boundary operators $\del_n: C_{n+1} \to C_n$, such that $\del_n \circ \del_{n+1} = 0$. The collections of all such objects will be denoted by $\Ch{\cat{Ab}}$.
\end{definition}

In other words a chain complex is the following diagram.
$$ \cdots \to C_4 \to C_3 \to C_2 \to C_1 \to C_0 $$

Of course we can make this more general by taking for example $R$-modules instead of abelian groups. We will later see which kind of algebraic objects make sense to use in this definition.



% \listoftodos
% \nocite{*}
% \bibliographystyle{alpha}
% \bibliography{references}	
\end{document}
