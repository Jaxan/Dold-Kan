\section{Homotopy}
\label{sec:Homotopy}

We've already seen homology in chain complexes. We can of course now translate this notion to simplicial abelian groups, by assigning a simplicial abelian group $X$ to $H_n(N(X))$. But there is a more general notion of homotopy for simplicial sets, which is also similar to the notion of homotopy in topology. We will define the notion of homotopy groups for simplicial sets.

When dealing with homotopy in a topological space $X$ we always need a base-point $\ast \in X$. This is also the case for homotopy in simplicial sets. We will notate the chosen base-point of a simplicial set $X$ with $\ast \in X_0$. Note that it is a $0$-simplex, but in fact the base-point is present in all sets $X_n$, because we can consider its degenerate simplices $s_0(\ldots(s_0(\ast))\ldots) \in X_n$, we will also denote these elements as $\ast$. Of course in our situation we are concerned about simplicial abelien groups, where there is an obvious choice for the base-point, namely $0$.

\todo{Htp: Do I want to define homotopy between maps?}

\begin{definition}
	Given a simplicial set $X$ with base-point $\ast$, we define $Z_n(X)$ to be the set of $n$-simplices with the base-point as boundary, i.e.:
	$$ Z_n(X) = \{ x \in X_n | d_i(x) = \ast \text{ for all } i \leq n \}. $$
	For two $n$-simplices $x, x' \in Z_n(X)$, we define $x \sim x'$ if there exists $y \in X_{n+1}$ such that:
	\begin{align}
		d_0(y) &= x \\
		d_1(y) &= x' \\
		d_i(y) &= \ast \text{ for all } i > 1.
	\end{align}
\end{definition}

Of course we would like $\sim$ to be an equivalence relation, however this is not true for all simplicial sets. For example there is in general no reason for symmetry, existence of a $1$-simplex $y$ from $x$ to $x'$ does not give us a $1$-simplex $y'$ from $x'$ to $x$. One can give an precise condition on when it is a equivalence relation, the so called Kan-condition. In our case of abelien groups, however, we can prove this directly.

\todo{Htp: Discuss/picturize Kan-condition?}

\begin{lemma}
	The relation $\sim$ as defined above is an equivalence relation on $Z_n(X)$.
\end{lemma}
\begin{proof}
	\todo{Htp: Make this a bit nicer}
	\emph{Reflexivity}. Let $x \in Z_n(X)$, define $y = s_0 x$. Now calculate $d_0 y = d_1 y = x$, because of the simplicial equations. And $d_i y = 0$ for all $i > 1$, because $x \in Z_n(X)$.

	\emph{Symmetry}. Let $x, x' \in Z_n(X)$ with $x \sim x'$. Let $y \in X_{n+1}$ such that $d_0 y = x$, $d_1 y = x'$ and $d_i y = 0$ for all $i > 1$. Define $y' = s_0 x + s_0 x' - y$, then by using linearity: $d_0 y' = x + x' - x = x'$ and $d_1 y' = x + x' - x' = x$. Again we get $d_i y' = 0$, because $x \in Z_n(X)$.

	\emph{Transitivity}. Let $x_0, x_1, x_2 \in Z_n(X)$ with $x_0 \sim x_1$ and $x_1 \sim x_2$. Let $x, z \in X_{n+1}$ such that ... Define $w = y + z - s_0 x_1$.
\end{proof}

\begin{definition}
	Given a simplicial abelian group $X$, we define the $n$-th homotopy group as:
	$$ \pi_n(X) = Z_n(X) / \sim. $$
\end{definition}

Note that this is an abelian group, because $Z_n(X)$ is a subgroup of $X_n$, and $\sim$ also defines a subgroup. It is relatively straight forward to prove that this definition coincides with the $n$-th homology group of the associated normalized chain complex.

\begin{lemma}
	For any simplicial abelian group $X$:
	$$ \pi_n(X) = H_n(N(X)). $$
\end{lemma}
\begin{proof}
	By writing out the definitions of the $n$-cycles and $n$-boundaries of the normalized chain complex, we see:
	\begin{align*}
		\ker(\del) &= \{ x \in N(X)_n \I \del(x) = 0 \} \\
			&= \{ x \in X_n \I d_i(x) = 0 \text{ forall } i > 0 \text{ and } d_0(x) = 0 \} \\
			&= \{ x \in X_n \I d_i(x) = 0 \text{ forall } i \leq n \} \\
			&= Z_n(X)
	\end{align*}
	\begin{align*}
		\im(\del) &= \{ \del(y) \I y \in N(X)_{n+1} \} \\
			&= \{ d_0 y \I y \in X_{n+1}, d_i(y) = 0 \text{ for all } i > 0 \} \\
			&= \{ x \in N(X)_n \I x \sim 0 \}
	\end{align*}
	So we see that $\pi_n(X) = Z_n(X) / \sim = \ker(\del) / \im(\del) = H_n(N(X))$.
\end{proof}

\begin{corollary}
	For a chain complex $C$ we have $H_n(C) \iso \pi_n(K(C))$
\end{corollary}
\begin{proof}
	By the established equivalence we have:
	$$ \pi_n(K(C)) \iso H_n(N(K(C))) \iso H_n(C). $$
\end{proof}


