\section{Constructions}
\label{sec:Constructions}

Comparing chain complexes and simplicial abelian groups, we see a similar structure. Both objects consists of a sequence of abelian groups, with maps in between. At first sight simplicial abelian groups have more structure, because there are maps in both directions. It is not clear how to make degeneracy maps given a chain complex, in fact it is already unclear how to define more maps (the face maps) out of one (the boundary one). Constructing a chain complex from a simplicial abelian group on the other hand seems doable.

\subsection{Unnormalized chain complex}
Given a simplicial abelian group $A$, we have a family of abelian groups $A_n$. We define a grouphomomorphism $\del_{n-1} : A_n \to A_{n-1}$ as:
$$\del_{n-1} = \delta^0 - \delta^1 + \ldots + (-1)^n \delta^n \text{ for every } n > 0.$$
\begin{lemma}
	Using $A_n$ as the family of abelian groups and the maps $\del_n$ as boundary maps gives a chain complex.
\end{lemma}
\begin{proof}
	We already have a collection of abelian groups together with maps, so the only thing to proof is $\del_n \circ \del_{n+1} = 0$.

	\todo{C: insert calculation with sums}

	So indeed this is a chain complex.
\end{proof}

This construction gives a functor $C : \sAb \to \Ch{\Ab}$\todo{C: prove this? Is it a adjunction?}. And in fact we already used it in the construction of the singular chaincomplex, where we defined the boundary maps as $\del(\sigma) = \sigma \circ \delta^0 - \sigma \circ \delta^1 + \ldots + (-1)^{n+1} \sigma \circ \delta^{n+1}$ (on generators). The terms $\sigma \circ \delta^i$ are the maps given by the $\mathbf{Hom}$-functor from $\Top$ to $\Set$, in fact this $\mathbf{Hom}$-functor can be used to get a functor $Sing : \Top \to \sSet$, applying the free abelain group pointwise give a functor $\Z^\ast : \sSet \to \sAb$, and finally using the functor $C$ gives the singular chain complex.
\todo{C: is this a nice thing to add?}

Let us investigate whether this functor can be used for our sought equivalence. For a functor from $\Ch{\Ab}$ to $\sAb$ we cannot simply take the same collection of abelian groups. This is due to the fact that the degenracy maps should be injective. This means that for a simplicial abelian group $A$, if we know $A_n$ is non-trivial, then all $A_m$ for $m > n$ are also non-trivial.

But for chain complexes it \emph{is} possible to have trivial abelian groups $C_m$, while there is a $n < m$ with $C_n$ non-trivial. Take for example the chain complex $ C = \ldots \to 0 \to 0 \to \Z $. Now if we would construct a (non-trivial) simplicial abelian group $K(C)$ from this chain complex, we now know that $K(C)_n$ is non-trivial for all $n \in \N$. This means that $C(K(C))_n$ is non-trivial for all $n \in \N$. For an equivalence we require a (natural) isomorphism: $C(K(C)) \tot{\iso} C$, this in particular means an isomorphism in each degree $n > 0$: $ 0 \neq C(K(C))_n \tot{\iso} C_n = 0 $, which is not possible. So the functor $C$, as defined as above, will not give us the equivalence we wanted, although it is a very nice functor.

\subsection{Normalized chain complex}
To repair this defect we should be more careful. Given a simplicial abelian group, simply taking the same collection for our chain complex will not work (as shown above). Instead we are after some ``smaller'' abelian groups, and in some cases the abelian groups should completely vanish (as in the example above).

Given a simplicial abelian group $A$, we define abelian groups $N(A)_n$ as:
$$ N(A)_n = \bigcap_{i=1}^{n} \ker(\delta^i : A_n \to A_{n-1}). $$
Now define grouphomomorphisms $\del : N(A)_n \to N(A)_{n-1}$ as:
$$ \del = \delta^0|_{N(A)_n}. $$
\begin{lemma}
	The function $ \del $ is well-defined. Furthermore $ \del \circ \del = 0 $, hence $N(A)$ is a chain complex.
\end{lemma}
\begin{proof}
	\todo{C: This is easy}
\end{proof}

\todo{C: As an example calculate $N(\Z[\Delta[0]])$}

\todo{C: The exciting part: $\Ch{\Ab} \to \sAb$}