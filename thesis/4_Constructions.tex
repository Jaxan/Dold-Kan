\section{The Dold-Kan correspondence}
\label{sec:Constructions}

Comparing chain complexes and simplicial abelian groups, one sees a certain similarity. Both concepts are defined as sequences of abelian groups with certain structure maps. At first sight simplicial abelian groups seem to have a richer structure. There are many face maps as opposed to only a single boundary homomorphism. Nevertheless, as we will show in this section, these two concepts give rise to equivalent categories.

\subsection{Unnormalized chain complex}
Given a simplicial abelian group $A$, we have a family of abelian groups $A_n$. For every $n>0$ we define a group homomorphism $\del_n : A_n \to A_{n-1}$
$$\del_n = d_0 - d_1 + \ldots + (-1)^n d_n.$$
\begin{lemma}
	Using $A_n$ as the family of abelian groups and the maps $\del_n$ as boundary maps gives a chain complex.
\end{lemma}
\begin{proof}
	We already have a collection of abelian groups together with maps, so the only thing to prove is $\del_{n-1} \circ \del_n = 0$. This can be done with a calculation.
	\begin{align*}
		\del_{n-1} \circ \del_n &= \sum_{i=0}^{n-1}  \sum_{j=0}^{n} (-1)^{i+j} d_i \circ d_j \\
			&\eqn{1} \sum_{i=0}^{n-1} \sum_{j=0}^{i} (-1)^{i+j} d_i \circ d_j + \sum_{i=0}^{n-1} \sum_{j=i+1}^{n} (-1)^{i+j} d_i \circ d_j \\
			&\eqn{2} \sum_{i=0}^{n-1} \sum_{j=0}^{i} (-1)^{i+j} d_i \circ d_j + \sum_{i=0}^{n-1} \sum_{j=i+1}^{n} (-1)^{i+j} d_{j-1} \circ d_i \\
			&\eqn{3} \sum_{i=0}^{n-1} \sum_{j=0}^{i} (-1)^{i+j} d_i \circ d_j - \sum_{i=0}^{n-1} \sum_{j=i}^{n-1} (-1)^{i+j} d_j \circ d_i \\
			&= \sum_{i=0}^{n-1} \sum_{j=0}^{i} (-1)^{i+j} d_i \circ d_j - \sum_{i=0}^{n-1} \sum_{j=0}^{i} (-1)^{i+j} d_i \circ d_j = 0
	\end{align*}
	In this calculation we did the following. We split the inner sum in two halves \refeqn{1} and we use the simplicial equations on the second sum \refeqn{2}. Then we do a shift of indices \refeqn{3}. By interchanging the roles of $i$ and $j$ in the second sum, we have two equal sums which cancel out. So indeed this is a chain complex.
\end{proof}

This construction defines a functor $M : \sAb \to \Ch{\Ab}$. And in fact we already used it in the construction of the singular chain complex, where we defined the boundary maps (on generators) as $\del(\sigma) = \sigma \circ d_0 - \sigma \circ d_1 + \ldots + (-1)^{n+1} \sigma \circ d_{n+1}$. We will briefly come back to this in Section~\ref{sec:Homotopy}.

Let us investigate whether this functor can be used for our sought equivalence. For a functor from $\Ch{\Ab}$ to $\sAb$ we cannot simply take the same collection of abelian groups. This is due to the fact that the degeneracy maps should be injective. This means that for a simplicial abelian group $A$, if we know $A_n$ is non-trivial, then all $A_m$ for $m > n$ are also non-trivial.

But for chain complexes it \emph{is} possible to have trivial abelian groups $C_m$, while there is a $n < m$ with $C_n$ non-trivial. Take for example the chain complex
$$ C = \ldots \to 0 \to 0 \to \Z. $$
Now if we would construct a (non-trivial) simplicial abelian group $K(C)$ from this chain complex, we now know that $K(C)_n$ is non-trivial for all $n \in \N$. This means that $M(K(C))_n$ is non-trivial for all $n \in \N$. For an equivalence we require a (natural) isomorphism: $M(K(C)) \tot{\iso} C$, this in particular means an isomorphism in each degree $n > 0$: $ 0 \neq M(K(C))_n \tot{\iso} C_n = 0 $, which is not possible. So the functor $M$, as defined as above, will not give us the equivalence we wanted, although it is a very nice functor.

\subsection{Normalized chain complex}
To repair this defect we should be more careful. Given a simplicial abelian group, simply taking the same collection for our chain complex will not work (as shown above). Instead we are after some ``smaller'' abelian groups, and in some cases the abelian groups should completely vanish (as in the example above).

Given a simplicial abelian group $A$, we define abelian groups $N(A)_n$ as
\begin{align*}
	N(A)_n &= \bigcap_{i=1}^{n} \ker(d_i : A_n \to A_{n-1}), \quad\text{n > 0} \\
	N(A)_0 &= A_0.
\end{align*}
Now define group homomorphisms $\del : N(A)_n \to N(A)_{n-1}$ as
$$ \del = d_0|_{N(A)_n}. $$
\begin{lemma}
	The function $ \del $ is well-defined. Furthermore $ \del \circ \del = 0 $.
\end{lemma}
\begin{proof}
	Let $x \in N(A)_n$, then $d_i \del(x) = d_i d_0(x) = d_0 d_{i+1}(x) = d_0 (0) = 0$ for all $i < n$. So indeed $\del(x) \in N(A)_{n-1}$, because in particular it holds for $i > 0$. Using this calculation for $i = 0$ shows that $\del \circ \del = 0$. This shows that $N(A)$ is a chain complex.
\end{proof}

The chain complex $N(A)$ is called the \emph{normalized chain complex} of $A$.
\begin{lemma}
	The above construction gives a functor $N: \sAb \to \Ch{\Ab}$. Furthermore $N$ is additive.
\end{lemma}
\begin{proof}
	Given a map $f: A \to B$ of simplicial abelian groups, we consider the restrictions
	$$ f_n |_{N(A)_n} : N(A)_n \to B_n. $$
	Because $f_n$ commutes with the face maps we get
	$$ d_i(f_n(x)) = f_{n-1}(d_i(x)) = 0, $$
	for $i>0$ and $x \in N(A)_n$. So the restriction also restricts the codomain, i.e. $f_n |_{N(A)_n} : N(A)_n \to N(B)_n$ is well-defined. Furthermore it commutes with the boundary operator, since $f$ itself commutes with all face maps. This gives functoriality $N(f): N(A) \to N(B)$.

	Let $f, g: A \to B$ be two maps, then
	$$ N(f+g) = (f+g)|_{N(A)} = f|_{N(A)} + g|_{N(A)} = N(f) + N(g). $$
	By recalling that in both categories addition of maps was defined pointwise, we have additivity of $N$.
\end{proof}

\begin{example}
	We will look at the normalized chain complex of $\Z[\Delta[0]]$. Recall that it looks like
	$$ \Z[\Delta[0]] :=
	\begin{tikzpicture}[baseline=-0.5ex]
	\matrix (m) [matrix of math nodes] { 
		\Z & \Z & \Z & \cdots \\
	}; 

	\foreach \r in {-5, 5} \draw [raise line=\r, <-] (m-1-1) -> (m-1-2);
	\foreach \r in {0} \draw [raise line=\r, ->] (m-1-1) -> (m-1-2);

	\foreach \r in {-10, 0, 10} \draw [raise line=\r, <-] (m-1-2) -> (m-1-3);
	\foreach \r in {-5, 5} \draw [raise line=\r, ->] (m-1-2) -> (m-1-3);

	\foreach \r in {-15, -5, 5, 15} \draw [raise line=\r, <-] (m-1-3) -> (m-1-4);
	\foreach \r in {-10, 0, 10} \draw [raise line=\r, ->] (m-1-3) -> (m-1-4);
	\end{tikzpicture}.$$
	where all face and degeneracy maps are identity maps. Clearly the kernel of $\id$ is the trivial group. So $N(\Z[\Delta[0]])_i = 0$ for all $i > 0$. In degree zero we are left with $N(\Z[\Delta[0]])_0 = \Z$. So we can depict the normalized chain complex by
	$$ N(\Z[\Delta[0]]) = \cdots \to 0 \to 0 \to \Z. $$
	So in this example we see that the normalized chain complex is really better behaved than the unnormalized chain complex given by $M(\Z[\Delta[0]]$.
\end{example}

To see what $N$ exactly does there are some useful lemmas. These lemmas can also be found in \cite[Chapter~VIII~1-2]{lamotke}, but in this thesis more detail is provided. Some corollaries are provided to give some intuition, or so summarize the lemmas, these results can also be found in \cite[Chapter~8.2-4]{weibel}. For the following lemmas let $X \in \sAb$ be an arbitrary simplicial abelian group and $n \in \N$. For these lemmas we will need the subgroups $D_n(X) \subset X_n$ of degenerate simplices, defined as:
$$ D_n(X) = \sum_{i=0}^n s_i(X_{n-1}). $$

\begin{lemma}
	\label{le:decomp1}
	For all $x \in X_n$ we have:
	$$ x = b + c,$$
	where $b \in N(X)_n$ and $c \in D_n(X)$.
\end{lemma}
\begin{proof}
	Define the subgroup $P_n^k = \{ x \in X_n \I d_i x = 0 \text{ for all } i > k\}$. Note that by definition we have
	$$ N(X)_n = P_n^0 \subseteq P_n^1 \subseteq \ldots \subseteq P_n^{n-1} \subseteq P_n^n = X_n. $$
	We will prove with induction that for any $k \leq n$ we can write $x \in X_n$ as $x = b + c$, with $b \in P_n^k$ and $c \in D_n(X)$. For $k = n$ the statement is clear, because we can simply write $x = x$, knowing that $x \in P_n^n = X_n$.

	Assume the statement holds for $k > 0$, we will prove it for $k-1$. So for any $x \in X_n$ we have $x = b + c$, with $b \in P_n^k$ and $c \in D_n(X)$. Now consider $b' = b - s_{k-1} d_k b$. Now clearly for all $i > k$ we have $d_i b' = 0$. For $k$ itself we can calculate
	$$ d_k(b') = d_k(b - s_{k-1} d_k b) = d_k b - d_k s_{k-1} d_k b = d_k b - d_k b = 0, $$
	where we used the equality $d_k s_{k-1} = \id$. So $b' \in P_n^{k-1}$. Furthermore we can define $c' = s_{k-1} d_k b + c$, for which it is clear that $c' \in D_n(X)$. Finally conclude that
	$$ x = b + c = b - s_{k-1} d_k b + s_{k-1} d_k b + c = b' + c',$$
	with $b' \in P_n^{k-1}$ and $c' \in D_n(X)$.

	Doing this inductively gives us $x = b + c$, with $b \in P_n^0 = N(X)_n$ and $c \in D_n(X)$, which is what we had to prove.
\end{proof}
\begin{lemma}
	\label{le:decomp2}
	For all $x \in X_n$, if $s_i x \in N(X)_{n+1}$, then $x = 0$.
\end{lemma}
\begin{proof}
	Using that $s_i x \in N(X)_{n+1}$ means $0 = d_{k+1} s_i x$ for any $k \geq 0$ and by using using the simplicial identity: $d_{i+1} s_i = \id$, we can conclude $x = d_{i+1} s_i x = 0$.
\end{proof}

The first lemma tells us that every $n$-simplex in $X$ can be decomposed as a sum of something in $N(X)$ and a degenerate $n$-simplex. The latter lemma assures that there are no degenerate $n$-simplices in $N(X)$. So this gives us:

\begin{corollary}
	\label{cor:NandD}
	$X_n = N(X)_n \oplus D_n(X)$
\end{corollary}

We can extend the above lemmas to a more general statement.

\begin{lemma}
	\label{le:decomp3}
	For all $x \in X_n$ we can write $x$ as:
	$$ x = \sum_\beta \beta^\ast (x_\beta), $$
	for certain $x_\beta \in N(X)_p$ and $\beta : [n] \epi [p]$.
\end{lemma}
\begin{proof}
	We will proof this using induction on $n$. For $n=0$ the statement is clear because $N(X)_0 = X_0$.

	Assume the statement is proven for $n$. Let $x \in X_{n+1}$, then from Lemma~\ref{le:decomp1} we see $x = b + c$. Note that $c \in D_n(X)$, in other words $c = \sum_{i=0}^{n-1} s_i c_i$, with $c_i \in X_n$. So with the induction hypothesis, we can write these as $c_i = \sum_\beta \beta^\ast c_{i, \beta}$, where the sum quantifies over $\beta: [n] \epi [p]$. Now $b$ is already in $N(X)_{n+1}$, so we can set $x_\id = b$, to obtain the conclusion.
\end{proof}
\begin{lemma}
	\label{le:decomp4}
	Let $\beta : [n] \epi [m]$ and $\gamma : [n] \epi [m']$ be two maps such that $\beta \neq \gamma$. Then we have $\beta^\ast(N(X))_m \cap \gamma^\ast(N(X))_{m'} = 0$.
\end{lemma}
\begin{proof}
	Note that $N(X)_i$ only contains non-degenerate $i$-simplices (and $0$). For $x \in \beta^\ast(N(X))_p \cap \gamma^\ast(N(X))_q$ we have $x = \beta^\ast y = \gamma^\ast y'$, where $y$ and $y'$ are non-degenerate. By Lemma~\ref{le:non-degenerate} we know that every $n$-simplex is \emph{uniquely} determined by a non-degenerate simplex and a surjective map. For $x \neq 0$ this gives a contradiction.
\end{proof}

Again the former lemma of these two lemmas proves the existence of a decomposition and the latter shows the uniqueness. So combining these gives:

\begin{corollary}
	\label{cor:decomp}
	For all $x \in X_n$ we can write $x = \sum_\beta \beta^\ast (x_\beta)$ in a unique way.
\end{corollary}

And by considering $X_n$ as a whole we get:

\begin{corollary}
	$X_n = \bigoplus_{[n] \epi [p]} N(X)_p$.
\end{corollary}

Using Corollary~\ref{cor:decomp} we can prove a nice categorical fact about $N$, which we will use later on.
\begin{lemma}
	The functor $N$ is fully faithful, i.e.:
	$$ N : \Hom{\sAb}{A}{B} \iso \Hom{\Ch{\Ab}}{N(A)}{N(B)} \quad A, B \in \sAb. $$
\end{lemma}
\begin{proof}
	First we prove that $N$ is injective on maps. Let $f: A \to B$ and assume $N(f) = 0$, for $x \in A_n$ we know $x = \sum_\beta \beta^\ast x_\beta$, so
	\begin{align*}
		f(x) &= \textstyle f(\sum_\beta \beta^\ast (x_\beta)) \\
			&= \textstyle \sum_\beta f(\beta^\ast (x_\beta)) \\ 
			&= \textstyle \sum_\beta \beta^\ast (f (x_\beta)) \\ 
			&= \textstyle \sum_\beta \beta^\ast (N(f) (x_\beta)) = 0,	
	\end{align*}
	where we used naturality of $f$ in the second step, and the fact that $x_\beta \in N(A)$ in the last step. We now see that $f(x) = 0$ for all $x$, hence $f = 0$. So indeed $N$ is injective on maps.

	Secondly we have to prove $N$ is surjective on maps. Let $g : N(A) \to N(B)$, define $f : A \to B$ as
	$$ f(x) = \sum_\beta \beta^\ast g(x_\beta), $$
	again we have written $x$ as $x = \sum_\beta \beta^\ast x_\beta$. Clearly $N(f) = g$.
\end{proof}

If we reflect a bit on why the functor $M$ was not a candidate for an equivalence, we see that $N$ does a better job. We see that $N$ leaves out all degenerate simplices, so it is more carefully chosen than $M$, which included everything. In fact, Corollary~\ref{cor:NandD} exactly tells us $M(X)_n = N(X)_n \oplus D_n(X)$.

\subsection{From $\Ch{\Ab}$ to $\sAb$}
For the other way around we actually get a functor for free, via abstract nonsense. Let $F : \sAb \to A$ be any functor, where $A$ is an abelian category. We are after a functor $G : A \to \sAb$, this means that if we are given $C \in A$, we are looking for a functor $G(C) : \DELTA^{op} \to \Ab$. Fixing $C$ in the second argument of the $\mathbf{Hom}$-functor gives: $\Hom{A}{-}{C} : A^{op} \to \Ab$, because $A$ is an abelian category. We see that the codomain of this functor already looks good, now if we have some functor from $\DELTA^{op}$ to $A^{op}$, we can precompose, to obtain a functor from $\DELTA^{op}$ to $\Ab$.

Now recall that we have a family of simplicial sets $\Delta[n]$, which are given by the functor $\Delta : \DELTA \to \sSet$. We can apply the free abelian group pointwise, which gives a functor $\Z^{\ast} : \sSet \to \sAb$. And finally we have our functor $F : \sAb \to A$. Composing these gives:
$$ F \Z^{\ast} \Delta : \DELTA \to A. $$
We can formally regard this functor as a functor from $\DELTA^{op}$ to $A^{op}$. Now combining this with the $\mathbf{Hom}$-functor gives:
$$ \Hom{A}{F \Z^{\ast} \Delta (-)}{C} : \DELTA^{op} \to \Ab. $$
Now this is a functor, because it is a composition of functors. Furthermore it is also functorial in the second argument, giving a functor
$$ \Hom{A}{F \Z^{\ast} \Delta (-)}{-} : A \to \sAb $$
where we are supposed to fill in the second argument first, leaving us with a simplicial abelian group.

Now we know that $\Ch{\Ab}$ is an abelian group and we have actually two functors $C, N : \sAb \to \Ch{\Ab}$, so we now have functors from $\Ch{\Ab} \to \sAb$. Of course we will be interested in the one using $N$. So we define the functor:
$$ K(C) = \Hom{\Ch{\Ab}}{N\Z^\ast\Delta[-]}{C} \in \sAb. $$
This definition is very abstract, but luckily we can also give a more explicit definition. By writing it out for low dimensions we see:

$$ K(C)_0 = \Hom{\Ch{\Ab}}{N\Z^\ast\Delta[0]}{C} = \Bigg\{
\begin{tikzpicture}[baseline=-0.5ex]
	\matrix (m) [matrix of math nodes, row sep=1em, column sep=1em] { 
		\cdots  & 0 & 0 & \Z  \\
		\cdots  & C_2 & C_1 & C_0 \\
	}; 

	\foreach \x in {1, 2}
		\foreach \i/\j in {1/2, 2/3, 3/4} \draw[->] (m-\x-\i) -- (m-\x-\j);
	
	\foreach \i/\j in {2/2, 3/1, 4/0} \draw[->] (m-1-\i) -- node {$f_\j$} (m-2-\i);
\end{tikzpicture}
\Bigg\} \iso C_0, $$
because for $f_1, f_2, \ldots$ there is now choice at all, and for $f_0 : \Z \to C_0$ we only have to choose an image for $1 \in \Z$. In the next dimension we see:

$$ K(C)_1 = \Hom{\Ch{\Ab}}{N\Z^\ast\Delta[1]}{C} = \Bigg\{
\begin{tikzpicture}[baseline=-0.5ex]
	\matrix (m) [matrix of math nodes, row sep=1em, column sep=1em] { 
		\cdots  & 0 & \Z & \Z^2  \\
		\cdots  & C_2 & C_1 & C_0 \\
	}; 

	\foreach \x in {1, 2}
		\foreach \i/\j in {1/2, 2/3, 3/4} \draw[->] (m-\x-\i) -- (m-\x-\j);
	
	\foreach \i/\j in {2/2, 3/1, 4/0} \draw[->] (m-1-\i) -- node {$f_\j$} (m-2-\i);
\end{tikzpicture}
\Bigg\} \iso C_1 \oplus C_0, $$

because again we can choose $f_1$ anyway we want, which gives us $C_1$. But then we are forced to choose $f_0(x, x) = \del(f_1(x))$ for all $x \in \Z$, so we are left with choosing an element $c \in C_0$ for defining $f(1,-1) = c$. Adding this gives $C_1 \oplus C_0$. This pattern can be continued and gives the following result:

\begin{proposition}
	For any chain complex $C$ we have $K(C)_n \iso \bigoplus_{[n] \epi [p]} C_p$.
\end{proposition}
\begin{proof}
	\todo{DK: proposition about $K$}
\end{proof}

\todo{DK: work out the following in more detail (especially the naturalities)}
\begin{theorem}
	$N$ and $K$ form an equivalence.
\end{theorem}
\begin{proof}
	Let $X$ be a simplicial abelian group. Consider $X_n$, by the abelian Yoneda lemma this is naturally isomorphic to $\Z^\ast[\Delta[n]] \to X$, which is by the fully faithfulness and additivity of $N$ naturally isomorphic to $N\Z^\ast[\Delta[n]] \to NX$. The latter is exactly the definition of $KNX$. So, by naturality in $n$, we have established $X \iso KNX$. Hence, by naturality in $X$, we have $\id \iso KN$.

	By the previous proposition we have $K(C)_n \iso \bigoplus_{[n] \epi [p]} C_p$. For the summands $C_p$ with $p < n$, we clearly see that $C_p \subset D_n(K(C))$, so $N$ gets rid of these. Then the only summand left is $C_n$ (with the surjection $\id : [n] \epi [n]$). So we see $NKC_n \iso C_n$, furthermore the boundary map is preserved. Hence $NKC \iso C$. And this was natural in $C$, so we get $NK \iso \id$.

	We now have established two natural isomorphisms $\id_\sAb \iso KN$ and $NK \iso \id_\Ch{\Ab}$. Hence we have an equivalence $\Ch{\Ab} \simeq \sAb$.
\end{proof}
