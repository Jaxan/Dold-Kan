\documentclass[12pt]{amsproc}

% a la fullpage
\usepackage{geometry}
\geometry{a4paper}
\geometry{twoside=false}

% Activate to begin paragraphs with an empty line rather than an indent
\usepackage[parfill]{parskip}
\setlength{\marginparwidth}{2cm}

\newtheorem{theorem}{Theorem}[section]
\newtheorem{definition}[theorem]{Definition}
\newtheorem{lemma}[theorem]{Lemma}

\usepackage{graphicx}
\usepackage{float}
\usepackage{amssymb}
\usepackage{color}
\usepackage{listings}

\newcommand{\id}{\text{id}}
\newcommand{\N}{\mathbb{N}}
\newcommand{\Z}{\mathbb{Z}}
\newcommand{\R}{\mathbb{R}}
\newcommand{\cat}[1]{\mathbf{#1}}
\newcommand{\Ch}[1]{\mathbf{Ch}(#1)}
\newcommand{\Hom}[3]{\mathbf{Hom}_{#1}(#2, #3)}

\newcommand{\iso}{\cong}
\newcommand{\tot}[1]{\xrightarrow{\,\,{#1}\,\,}}
\newcommand{\eps}{\varepsilon}
\newcommand{\I}{\,\mid\,}
\newcommand{\then}{\Rightarrow}
\newcommand{\inject}{\hookrightarrow}
\newcommand{\del}{\partial}
\newcommand{\nsubgrp}{\trianglelefteq}

% relative to the one who includes us :(
\graphicspath{ {../images/} }

\newcommand{\todo}[1]{
	\addcontentsline{tdo}{todo}{\protect{#1}}
	$\ast$ \marginpar{\tiny $\ast$ #1}
}
\makeatletter
	\newcommand \listoftodos{\section*{Todo list} \@starttoc{tdo}}
	\newcommand\l@todo[2]{
		\par\noindent \textit{#2}, \parbox{10cm}{#1}\par
	}
\makeatother


\begin{document}

% For basic categorical picture of simplicial objects
% $$ [0] \to [1] \to [2] \to [3] \to \ldots $$
% $$\delta_i: [n] \to [n+1], k \mapsto \begin{cases} k & \text{if } k < i;\\ k+1 & \text{if } k \geq i. \end{cases} \hspace{0.5cm} 0 \leq i \leq n+1$$
% $$\sigma_i: [n+1] \to [n], k \mapsto \begin{cases} k & \text{if } k \leq i;\\ k-1 & \text{if } k > i. \end{cases} \hspace{0.5cm} 0 \leq i \leq n$$
% $$ A_0 \to A_1 \to A_2 \to A_3 $$

% For geometric picture of simplicial objects
% $$ 0 \tot{\delta_0} 1 \tot{\delta_1} 2 \tot{\delta_2} 3 \tot{\delta_3} \cdots $$

% For the pictures in the presentation (singular chain complex)
$$ \cdots \tot{\del_2} C_2 \tot{\del_1} C_1 \tot{\del_0} C_0 $$
\reflectbox{\rotatebox[origin=c]{90}{\large $=$}}
$$ + - \mapsto $$
$$ \{ \} $$

\end{document}
